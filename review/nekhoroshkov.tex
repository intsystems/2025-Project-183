\documentclass[a4paper,12pt]{article}
\usepackage[T2A,T1]{fontenc}
\usepackage[utf8]{inputenc}
\usepackage[russian,english]{babel}
\usepackage{enumitem}
\usepackage{geometry}
\geometry{left=2cm,right=2cm,top=2cm,bottom=2cm}

\begin{document}

\pagestyle{empty}

\begin{center}
  {\Large\textbf{Review}}\\[1em]
  {\large\textit{Detecting Manual Alterations in Biological Image Data Using Contrastive Learning and Pairwise Image Comparison}}\\[0.5em]
  {\selectlanguage{russian}Nekhoroshkov G.}
\end{center}

\vspace{1em}

\begin{enumerate}[leftmargin=*, label=\arabic*.]
  \item Suggest increasing the font size of the axis tick labels on the AUC-ROC plots.
  \item Add references when mentioning figures in the text to facilitate easier navigation.
  \item Make the GitLab link clickable.
  \item Consider adding a section on future work and limitations; this would simplify any subsequent explorations.
  \item There is no discussion section, as indicated in the outline. I suggest adding one, since the conclusions about the 
  method and the chosen fixed hyperparameters in Section 5 are currently unclear.
  \item Half of the bibliography is not cited in the article. I recommend incorporating more of these references into the 
  related work section to give readers deeper insight and clearer connections.
  \item For clarity --- and because the code is already available on GitHub --- keep only the theoretical algorithm descriptions 
  in the main text.
  \item Not all cited articles include links to their arXiv pages or to the code. Please add these links in the bibliography 
  for easier reference.
\end{enumerate}

\vspace{1em}

\noindent\textbf{Overall opinion.} The problem statement and the motivation for this exploration are presented clearly. 
The author has surveyed numerous prior works and discussed their relevance thoughtfully. The theoretical section is easy to 
follow, and the descriptions of the base algorithms helped me understand both the core idea and the experimental setup. 
The results are displayed graphically, providing a strong sense of their effectiveness and potential applications. Once the 
minor fixes above are implemented, this paper could represent a significant advance in machine-learning–based medical 
detection technologies!

\vspace{2em}

\noindent Reviewer:\\
Nikitin A.

\end{document}
