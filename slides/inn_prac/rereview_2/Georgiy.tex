\documentclass[11pt]{article}

\usepackage[utf8]{inputenc}
\usepackage[T2A]{fontenc}
\usepackage[russian]{babel}
\usepackage{amsmath,amssymb}
\usepackage[left=2cm,right=2cm,top=2cm,bottom=2cm]{geometry}

\begin{document}

\pagestyle{empty}

\begin{center}
    \subsection*{Ответ на замечания рецензента Нехорошкова Георгия Сергеевича}
\end{center}

% Положительные замечания
\begin{quote}
    \textbf{Рецензент:}
    «Автору удалось чётко оформить новый шаг исследования: введение $\Delta$-функции и применение Монте-Карло аппроксимации
    придаёт работе практическую ценность и понятный алгоритм оценки «точки насыщения» данных. Особенно ценно, что
    предложенный подход позволяет заранее планировать объём сбора данных и существенно экономить вычислительные ресурсы.
    Низкоразмерная визуализация изменений ландшафта и эмпирическая верификация теоретических границ усиливают доверие к
    результатам. Вторая версия презентации демонстрирует значительный прогресс в работе: добавлены конкретные вычислительные
    приёмы, практические рекомендации и визуализация метода. Мне очень понравилась данная работа, она выглядит готовой к
    расширенному экспериментальному этапу на более сложных задачах.»
\end{quote}

\paragraph{Ответ автора.}
Большое спасибо за высокую оценку проделанной работы и за поддержку идей проекта. Рад, что алгоритм оценки порога \(K\), визуализация ландшафта и эмпирическая проверка теории выглядят убедительно и информативно.

% Критические замечания

\begin{quote}
    \textbf{Рецензент:}
    «По-прежнему эксперименты ограничены MNIST и Fashion-MNIST — стоило бы протестировать метод на более сложных и разнообразных данных или современных архитектурах (например, CIFAR-10 или ResNet).»
\end{quote}

\paragraph{Ответ автора.}
Согласен, что расширение экспериментальной базы повысит практическую значимость метода. В ближайшем расширенном этапе работы планируется:
\begin{itemize}
    \item Тестирование критерия \(\Delta_k\) и порога \(K\) на датасетах CIFAR-10 и CIFAR-100;
    \item Применение к современным архитектурам (ResNet, DenseNet), где предварительные эксперименты уже показывают аналогичную динамику уменьшения \(\Delta_k\).
\end{itemize}

\begin{quote}
    \textbf{Рецензент:}
    «Также не до конца ясно, как именно выбирается распределение \(p(w)\) для вычисления \(\Delta_k\) и насколько чувствительны результаты к этому выбору.»
\end{quote}

\paragraph{Ответ автора.}
В настоящей версии работы мы используем гауссово распределение
\[
    p(w) = \mathcal{N}\bigl(w^*,\,\sigma^2 I\bigr),
\]
где \(w^*\) — точка минимума потерь, а \(\sigma\) подбирается из соображений масштаба гессиана, то есть \(\sigma^2 \approx \tfrac{1}{\lambda_{\max}(H(w^*))}\).
\begin{quote}
    \textbf{Рецензент:}
    «Было бы полезно видеть сравнение с другими существующими метриками «точки насыщения» данных.»
\end{quote}

\paragraph{Ответ автора.}
Полностью согласен, что сравнение с альтернативными подходами укрепит аргументацию. В расширенной версии будут добавлены:
\begin{itemize}
    \item Сравнение с порогами, основанными на уменьшении нормы градиента \(\|\nabla L_k(w)\|\);
    \item Таблица корреляций между нашим порогом \(K\) и порогами других метрик на одном и том же наборе экспериментов.
\end{itemize}
Это позволит продемонстрировать преимущества предложенного \(\Delta_k\)-подхода в плане стабильности и согласованности.

\paragraph{Заключение.}
Благодарю за ценные замечания — они помогут сделать статью более полной и убедительной. Все перечисленные доработки будут реализованы в финальной версии.

\end{document}
