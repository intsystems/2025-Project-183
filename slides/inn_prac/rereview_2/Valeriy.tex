\documentclass[11pt]{article}

\usepackage[utf8]{inputenc}
\usepackage[T2A]{fontenc}
\usepackage[russian]{babel}
\usepackage{amsmath,amssymb}
\usepackage[left=2cm,right=2cm,top=2cm,bottom=2cm]{geometry}

\begin{document}

\pagestyle{empty}

\begin{center}
    \subsection*{Ответ на замечания рецензента Леванова Валерия Дмитриевича}
\end{center}

% Положительные замечания
\begin{quote}
    \textbf{Рецензент:}
    «Автор чётко обозначает проделанную работу, план действий и ожидаемые результаты. Работа интересна с точки зрения практического применения, так как оптимизация использования вычислительных ресурсов — очень важная тема. Большим плюсом является чёткая структура доклада, но мне кажется, что тут много лишних слайдов. Также не очень понятно, какие дальнейшие действия будут предприняты в работе.»
\end{quote}

\paragraph{Ответ автора.}
Благодарю за содержательный отзыв и высокую оценку практической значимости исследования. Рад, что структура доклада и изложение плана работ показались ясными и полезными.

\begin{itemize}
    \item \textbf{По поводу структуры и объёма слайдов.}
          % Уточнение про лишние слайды
          Я учту замечание о избыточности некоторых слайдов. В окончательном варианте презентации (а также в тексте статьи) будут:
          \begin{itemize}
              \item Сокращены вспомогательные иллюстрации: объединены близкие по смыслу графики $\Delta_k$ для разных итераций в единые панели;
              \item Упрощены избыточные блоки с описанием базовых свойств гессиана, перенесённые в приложение;
              \item Чётко выделен основный поток изложения без второстепенных технических деталей.
          \end{itemize}

    \item \textbf{О дальнейших шагах.}
          % План будущей работы
          В финальной версии статьи будет отдельный раздел «Планы и будущие работы», где подробно описаны:
          \begin{itemize}
              \item Расширение экспериментов на датасетах CIFAR-10/100 и современных архитектурах (ResNet, DenseNet);
              \item Разработка онлайн-версии метода с динамическим отслеживанием числа собственных векторов $d$ и оценкой изменения ландшафта $\Delta_k$;
              \item Уточнение теоретических границ сходимости через неасимптотические оценки и учёт высших моментов распределения данных;
              \item Улучшение визуализации: зависимость точности модели от выбранного порога $\Delta$ и открытый релиз кода.
          \end{itemize}
\end{itemize}

Эти доработки позволят сделать работу более компактной, ясной и завершённой, а также дадут читателям ясное представление о дальнейших перспективах исследования. Еще раз благодарю за ценные замечания!

\end{document}
