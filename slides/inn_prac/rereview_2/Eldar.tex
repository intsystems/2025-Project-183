\documentclass[11pt]{article}

\usepackage[utf8]{inputenc}
\usepackage[T2A]{fontenc}
\usepackage[russian]{babel}
\usepackage{amsmath,amssymb}
\usepackage[left=2cm,right=2cm,top=2cm,bottom=2cm]{geometry}

\begin{document}

\pagestyle{empty}

\begin{center}
    \subsection*{Ответ на замечания рецензента Хузина Эльдара Руслановича}
\end{center}

% Положительные замечания
\begin{quote}
    \textbf{Рецензент:}
    «Структура доклада: цель, имеющиеся работы, постановка проблемы, описание эксперимента и его результатов и в завершении результаты всей работы — хорошая, качественная структура, что является безусловным плюсом. Также хочу отметить понятные и хорошо дополняющие визуализации графиков и таблиц данных. По научной части: новизна — есть; формализм — есть; проверка на реальных данных — тоже есть — это классно. Сильной стороной работы считаю её потенциал для уменьшения порога входа начинающим специалистам — т.е. быть может работа даст стимул к созданию небольших, но качественных датасетов. Интересен ход с использованием Гессиана, приятно видеть, как применяется недавно пройденный материал.»
\end{quote}

\paragraph{Ответ автора.}
Большое спасибо за позитивную оценку структуры доклада, визуализаций и научного содержания. Рад, что идея порога «точки насыщения» и метод, основанный на Гессиане, воспринимаются как полезный инструмент для практиков с ограниченными ресурсами.

% Замечание 1: верстка
\begin{quote}
    \textbf{Рецензент:}
    «Из недостатков разве что могу отметить некоторую неудачность вёрстки — в основном в части переноса формул на новые строки; отсутствие подписей на графиках.»
\end{quote}

\paragraph{Ответ автора.}
Согласен, что аккуратная верстка важна для восприятия. В финальной версии:
\begin{itemize}
    \item Все рисунки и графики получат подписанные оси, легенды и развернутые подписи (\verb|\caption|), чтобы читателю не требовалось дополнительное объяснение.
\end{itemize}

% Замечание 2: простые датасеты
\begin{quote}
    \textbf{Рецензент:}
    «Минусы: простые датасеты (хотя, с другими было бы сложно проводить сам анализ)…»
\end{quote}

\paragraph{Ответ автора.}
Мы планируем расширить экспериментальную базу:
\begin{itemize}
    \item Добавить эксперименты на CIFAR-10/100 и на наборе Tiny ImageNet.
    \item Протестировать критерий \(\Delta_k\) и порог \(K\) на современных архитектурах (ResNet, DenseNet).
    \item Сравнить динамику \(\Delta_k\) для разных типов данных (изображения, текст) и оценить универсальность вывода.
\end{itemize}

% Замечание 3: сложность вычислений гессиана
\begin{quote}
    \textbf{Рецензент:}
    «Практическое ограничение применения подхода (из-за сложности расчётов Гессиана).»
\end{quote}

\paragraph{Ответ автора.}
Для масштабирования метода мы используем эффективные приёмы:
\begin{itemize}
    \item Вычисление произведения гессиана на вектор через метод Пейкана–Ритса по автодифференцированию, что даёт \(\mathcal{O}(N)\) стоимость одной операции вместо \(\mathcal{O}(N^2)\).
    \item Подбор только \(d\ll N\) ведущих собственных векторов с помощью Ланцоша, что сокращает расходы до \(\mathcal{O}(dN)\) на одну оценку \(\Delta_k\).
    \item Исследование стохастических аппроксимаций гессиана через мини-бэтчи, позволяющих контролировать точность аппроксимации.
\end{itemize}
Эти приёмы делают возможным применение метода к сетям с десятками миллионов параметров.

% Замечание 4: архитектурная сложность и применимость анализа
\begin{quote}
    \textbf{Рецензент:}
    «…при использовании хорошего математического аппарата для анализа, у нас имеется сложная структура на вход, влияние свойств которой могут плохо поддаваться учёту.»
\end{quote}

\paragraph{Ответ автора.}
Наш подход априори не зависит от архитектуры: мы исследуем ландшафт функции потерь в окрестности найденной точки \(w^*\). Свойства входных данных и архитектуры отражаются в гессиане \(H(w^*)\). В доработанной статье будет:
\begin{itemize}
    \item Анализ чувствительности собственных значений \( \{\lambda_i\} \) к изменениям структуры сети (сравнение CNN vs MLP).
    \item Обсуждение влияния особенностей архитектуры (свёртки, нормализационные слои) на спектр гессиана.
    \item Иллюстративный пример: графики распределения \(\lambda_i\) для ResNet-18 vs простой ConvNet.
\end{itemize}

\paragraph{Заключение.}
Ещё раз благодарю за детальные замечания. Все предложенные улучшения будут реализованы в окончательной версии работы, что сделает её более читаемой, универсальной и практически применимой.

\end{document}
