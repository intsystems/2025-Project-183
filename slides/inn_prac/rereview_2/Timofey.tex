\documentclass[11pt]{article}

\usepackage[utf8]{inputenc}
\usepackage[T2A]{fontenc}
\usepackage[russian]{babel}
\usepackage{amsmath,amssymb}
\usepackage[left=2cm,right=2cm,top=2cm,bottom=2cm]{geometry}

\begin{document}

\pagestyle{empty}

\begin{center}
    \subsection*{Ответ на замечания рецензента Кондрашова Тимофея}
\end{center}

% Положительные замечания
\begin{quote}
    \textbf{Рецензент:}
    ``Данная работа привлекает к себе внимание амбициозностью поставленной задач. Его доклад оставил положительное впечатление. Представленная презентация аккуратно оформлена и четко структурирована. Рассказ был насыщен на технические детали, но при этом понятен. Также помогает восприятию большое количество графиков и другой визуализации результатов.''
\end{quote}

\paragraph{Ответ автора.}
Благодарю за высокую оценку проделанной работы и подачи материала. Мы рады, что структура доклада, технические детали и визуализация результатов оказались понятными и информативными.

% Критическое замечание
\begin{quote}
    \textbf{Рецензент:}
    ``К минусам я бы лишь отнес то, что автор не включил дальнейшие шаги в свой доклад.''
\end{quote}

\paragraph{Ответ автора.}
Согласен с замечанием, в окончательной версии статьи будет добавлен раздел \emph{«Планы и будущие работы»}, в котором подробно изложены следующие направления развития:

\begin{itemize}
    \item \textbf{Расширение экспериментальной проверки на больших и разнообразных датасетах.}
    \begin{itemize}
        \item Исследование критерия \(\Delta_k\) и порога \(K\) на CIFAR-10/CIFAR-100.
        \item Тестирование на современных архитектурах (ResNet, DenseNet).
    \end{itemize}

    \item \textbf{Адаптация метода к потоковым (streaming) данным.}
    \begin{itemize}
        \item Разработка онлайн-версии алгоритма с динамическим обновлением числа компонент \(d\) и оценкой \(\Delta_k\).
    \end{itemize}

    \item \textbf{Уточнение теоретических границ сходимости.}
    \begin{itemize}
        \item Получение неасимптотических оценок скорости \(\Delta_k \to 0\) в зависимости от распределения входных данных.
        \item Выведение более точных оценок через учёт высших моментов и корреляций между собственными значениями гессиана.
    \end{itemize}

    \item \textbf{Улучшение визуализации и открытый релиз кода.}
    \begin{itemize}
        \item Дополнительные графики: зависимость точности модели от значения порога \(\Delta\).
        \item Публикация репозитория с реализацией Monte Carlo в низкоразмерном подпространстве и скриптов для reproducing экспериментов.
    \end{itemize}
\end{itemize}

Таким образом, раздел \emph{«Планы и будущие работы»} сделает статью более завершённой и позволит читателям ознакомиться с перспективами развития метода. Еще раз благодарю за полезный совет!

\end{document}
