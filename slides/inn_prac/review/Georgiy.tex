\documentclass[11pt]{article}

\usepackage[utf8]{inputenc}
\usepackage[T2A]{fontenc}
\usepackage[russian]{babel}
\usepackage{amsmath,amssymb}

\begin{document}

% ==============================
% Заголовок и идентификация отзыва
% ==============================
\begin{center}
    {\Large \textbf{Отзыв на доклад}}\\[1em]
    {\large \textbf{Название доклада:} Обнаружение фальсификаций в биологических изображениях с помощью контрастивного обучения}\\[0.5em]
    {\large \textbf{Автор доклада:} Нехорошков Георгий Сергеевич}\\[0.5em]
    {\large \textbf{Автор отзыва:} Никитин Артем Анатольевич}\\[2em]
\end{center}

% ==============================
% Секция 1: Аннотация (Abstract)
% ==============================
% Требования к аннотации:
% - Объём: от 70 до 150 слов (примерно 450–1050 символов без пробелов).
% - Текст должен состоять из нескольких предложений.
% - Необходимые элементы (при наличии в докладе):
%   * Формулировка задачи.
%   * Научная мотивация (зачем заниматься данной задачей).
%   * Обзор предыдущих результатов (наличие или отсутствие литературного обзора).
%   * Перспективы: какие результаты ожидается достичь к концу семестра.
%   * Описание предложенных методов решения задачи (конкретные алгоритмы, теоремы, планы) 
%     и упоминание, имеются ли промежуточные результаты.
\section*{Аннотация}
В докладе рассматривается проблема фальсификаций и заимствований биологических изображений,
что приводит к искажению научных данных в биомедицинских исследованиях. Проблема состоит в том,
что существующие модели контрастивного обучения, такие как Barlow Twins, CLIP и SimCLR, обученные на обычных изображениях,
демонстрируют ухудшение метрик на биологических датасетах. Автор предлагает модифицированную архитектуру, основанную на
модели Barlow Twins, с использованием предобученного энкодера и специализированного прожектора,
адаптированных для биологических снимков. Ожидается, что модель позволит улучшить показатели точности,
обеспечивая более надёжное обнаружение манипуляций с изображениями.

% ==============================
% Секция 2: Критический комментарий (Review)
% ==============================
% Требования к критическому комментарию:
% - Объём: от 50 до 150 слов (примерно 350–1050 символов без пробелов).
% - Текст должен состоять из нескольких предложений.
% - Обсудите сильные и слабые стороны доклада, как с точки зрения структуры и презентации 
%   (ясность изложения, визуальные материалы, ответы на вопросы), так и с точки зрения научного содержания 
%   (формулировка задачи, применённые методы, метрики и т.п.).
% - Отметьте, почему лично вам эта работа кажется интересной и в чем её научная ценность.
% - При оценке учитывайте вашу компетенцию и комментируйте те аспекты работы, которые вам понятны.
\section*{Комментарий}
Доклад представляет собой исследование, в котором автор ясно формулирует проблему фальсификаций биологических изображений и 
предлагает метод ее решения. Отмечу отдельно применение модифицированной архитектуры Barlow Twins,
адаптированной для специфических биомедицинских данных, что позволяет ожидать улучшения метрик. Презентация на семинаре
была изложена понятно, основной акцент был сделан на актуальности проблемы и практической применимости метода.
Единственным недостатком я отмечу чересчур краткое описание предварительных экспериментальных результатов,
но можно ожидать существенных продвижений по окончании работы автором. Лично мне доклад кажется интересным
благодаря перспективности использования современных методов контрастивного обучения в биомедицинских исследованиях.

\end{document}
