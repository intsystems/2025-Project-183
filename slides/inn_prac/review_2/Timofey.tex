\documentclass[11pt]{article}

\usepackage[utf8]{inputenc}
\usepackage[T2A]{fontenc}
\usepackage[russian]{babel}
\usepackage{amsmath,amssymb}

\begin{document}

% ==============================
% Заголовок и идентификация отзыва
% ==============================
\begin{center}
    {\Large \textbf{Отзыв на доклад}}\\[1em]
    {\large \textbf{Название доклада:} Локализация по графу сцены (Scene graph place recognition)}\\[0.5em]
    {\large \textbf{Автор доклада:} Кондрашов Тимофей}\\[0.5em]
    {\large \textbf{Автор отзыва:} Никитин Артем Анатольевич}\\[2em]
\end{center}

% ==============================
% Секция 1: Аннотация
% ==============================
\section*{Аннотация}
В докладе представлена задача локализации робота в помещении с помощью графов сцены, где вершины—объекты, а
рёбра отражают пространственные и семантические связи. Актуальность обусловлена необходимостью повышения точности и
надёжности локализации при использовании RGBD-данных и семантической сегментации. В обзоре рассматриваются методы
SceneGraphLoc, демонстрирующие R@1 до 97\% на пяти комнатах, а также соседние подходы (LiDAR GAT, Multiview Scene Graph).
Для практической реализации автор интегрирует библиотеку OpenPlaceRecognition с датасетом 3RScan и изучает альтернативу
DINOv2 в SegVLAD—«мешок запросов» (Bag of Queries). Ожидается повышение метрик поиска места и устойчивость к
изменению окружения.

% ==============================
% Секция 2: Критический комментарий
% ==============================
\section*{Критический комментарий}
Доклад логично структурирован: чёткое определение задачи, обзор ключевых публикаций и анализ базового метода SceneGraphLoc.
Особенно ценным является интеграция с OpenPlaceRecognition и практическая работа с 3RScan. Эксперименты по замене DINOv2
на Bag of Queries и введение весов сегментов выглядят многообещающе. Недостаток—пока нет количественных результатов новых
вариантов, хотелось бы увидеть сравнение до и после изменений. Тема представляется интересной в контексте развития
графовых методов для place recognition; лично я особенно ценю практическую направленность и возможность применения в
реальных автономных системах.

\end{document}
