\documentclass[11pt]{article}

\usepackage[utf8]{inputenc}
\usepackage[T2A]{fontenc}
\usepackage[russian]{babel}
\usepackage{amsmath,amssymb}

\begin{document}

% ==============================
% Заголовок и идентификация отзыва
% ==============================
\begin{center}
    {\Large \textbf{Отзыв на доклад}}\\[1em]
    {\large \textbf{Название доклада:} Detecting Manual Alterations in Biological Image Data Using Contrastive Learning and Pairwise Image Comparison}\\[0.5em]
    {\large \textbf{Автор доклада:} Нехорошков Георгий Сергеевич}\\[0.5em]
    {\large \textbf{Автор отзыва:} Никитин Артем Анатольевич}\\[2em]
\end{center}

% ==============================
% Секция 1: Аннотация
% ==============================
\section*{Аннотация}
Доклад посвящён задаче обнаружения мануальных модификаций биологических изображений, возникающих при фальсификации 
и несанкционированном заимствовании научных данных. Актуальность обусловлена необходимостью защиты оригинальности снимков, 
так как современные SSL-модели (Barlow Twins, SimCLR, CLIP) демонстрируют слабую точность в биомедицинских датасетах. 
Автор предлагает адаптированную архитектуру на базе ResNet-50 с проекторами и линейным классификатором, обучаемыми 
последовательно и параллельно на искусственно модифицированных парах. Используется оптимизатор AdamW с понижающимся 
шагом и специально собранный датасет из 700 изображений. Ожидается существенное улучшение метрик accuracy, F1-score, 
precision, recall и AUC-ROC по сравнению с исходным Barlow Twins.

% ==============================
% Секция 2: Критический комментарий
% ==============================
\section*{Критический комментарий}
Доклад убедительно формулирует проблему устойчивого детектирования подделок биологических снимков и предлагает 
практическое SSL-решение. Сильной стороной является ясное сравнение двух режимов обучения — последовательного и 
параллельного — а также подробное описание архитектуры с ResNet-50 и прожектором. Отдельно стоит отметить тщательный 
подбор датасета и настройку оптимизатора AdamW. Хотелось бы больше информации о распределении типов атак и о чувствительности 
к каждой из них. Тема представляется интересной благодаря прямой применимости верификации результатов биомедицинских 
исследований.

\end{document}
