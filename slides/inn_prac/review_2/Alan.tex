\documentclass[11pt]{article}

\usepackage[utf8]{inputenc}
\usepackage[T2A]{fontenc}
\usepackage[russian]{babel}
\usepackage{amsmath,amssymb}

\begin{document}

% ==============================
% Заголовок и идентификация отзыва
% ==============================
\begin{center}
    {\Large \textbf{Отзыв на доклад}}\\[1em]
    {\large \textbf{Название доклада:} Оценки чисел Рамсея для произвольных последовательностей графов}\\[0.5em]
    {\large \textbf{Автор доклада:} Беремкулов Алан}\\[0.5em]
    {\large \textbf{Автор отзыва:} Никитин Артем Анатольевич}\\[2em]
\end{center}

% ==============================
% Секция 1: Аннотация
% ==============================
\section*{Аннотация}
В докладе рассмотрены обобщения классических чисел Рамсея на произвольные последовательности графов. Проблема заключается
в узости традиционного определения и отсутствии точных оценок, например для \(R(5,5)\). Автор вводит новые параметры
\(R_{\min}(\{G_n\},k)\) и \(R_{\max}(\{G_n\},k)\), которые при соответствующих выборе графов сводятся к классическим
числам Рамсея. Представлены два основных результата: точные оценки \(R_{\max}\) для графов Джонсона (Теорема 1) и
асимптотические границы для произвольных семейств (Теорема 2). Также приведены начальные оценки \(R_{\min}\) и
обсуждаются дальнейшие направления по поиску новых семейств графов и уточнению оценок.

% ==============================
% Секция 2: Критический комментарий
% ==============================
\section*{Критический комментарий}
Доклад отличается чёткой постановкой задачи и строгой математической структурой. Сильной стороной является введение
обобщённых определений \(R_{\min}\) и \(R_{\max}\), расширяющих классическую теорию Рамсея. Презентация логична и
последовательно развивает идеи, однако хотелось бы увидеть конкретные численные примеры или иллюстрации оценок, что помогло 
бы оценить практическую значимость результатов. Было бы полезно продемонстрировать применение теорем на синтетических или 
известных графах. Лично мне тема близка благодаря её связям с дискретной математикой и потенциалу для дальнейших теоретических 
исследований.

\end{document}
