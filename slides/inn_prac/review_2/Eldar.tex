\documentclass[11pt]{article}

\usepackage[utf8]{inputenc}
\usepackage[T2A]{fontenc}
\usepackage[russian]{babel}
\usepackage{amsmath,amssymb}

\begin{document}

% ==============================
% Заголовок и идентификация отзыва
% ==============================
\begin{center}
    {\Large \textbf{Отзыв на доклад}}\\[1em]
    {\large \textbf{Название доклада:} Comparative Analysis of Data-Driven Approaches for Hydrological Forecasting}\\[0.5em]
    {\large \textbf{Авторы доклада:} Eldar Khuzin, Ivan Novikov, Dmitrii Abramov}\\[0.5em]
    {\large \textbf{Автор отзыва:} Никитин Артем Анатольевич}\\[2em]
\end{center}

% ==============================
% Секция 1: Аннотация
% ==============================
% Объём: 70–150 слов, несколько предложений, включает:
% - формулировку задачи и мотивацию;
% - обзор предыдущих подходов;
% - метод и план исследований;
% - ожидаемые результаты.
\section*{Аннотация}
В статье рассматривается проблема эквифинальности в data-driven гидрологическом моделировании, когда разные наборы признаков 
или архитектуры моделей демонстрируют сходную эффективность прогнозов. Актуальность работы определяется необходимостью 
выбора оптимальной модели с учётом точности, интерпретируемости и вычислительных затрат. Авторы анализируют влияние 
статических физико-географических характеристик водосбора на точность прогнозов, сравнивая классические методы, рекуррентные 
нейронные сети и трансформеры на данных по бассейнам рек Евразии. С помощью анализа важности признаков и поведения моделей 
выявляются доминирующие предикторы и определяется баланс между сложностью архитектуры и устойчивостью моделей при 
эквифинальности.

% ==============================
% Секция 2: Критический комментарий
% ==============================
% Объём: 50–150 слов, несколько предложений, включает:
% - сильные и слабые стороны;
% - личную заинтересованность.
\section*{Критический комментарий}
Работа чётко формулирует ключевую проблему эквифинальности и предлагает комплексный сравнительный анализ подходов, что важно 
для практического применения в гидрологии. Особенно ценю исследование влияния статических физико-географических признаков, 
повышающее интерпретируемость результатов. Сильной стороной является сочетание разных моделей и метрик при оценке 
вычислительной эффективности. Вместе с тем эксперименты ограничены бассейнами Евразии, что может сужать обобщаемость 
выводов. Тема представляется мне интересной в контексте оптимизации ресурсов и повышения надёжности моделей гидрологического 
прогнозирования.

\end{document}
