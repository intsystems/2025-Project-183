\documentclass[11pt]{article}

\usepackage[utf8]{inputenc}
\usepackage[T2A]{fontenc}
\usepackage[russian]{babel}
\usepackage{amsmath,amssymb}

\begin{document}

% ==============================
% Заголовок и идентификация отзыва
% ==============================
\begin{center}
    {\Large \textbf{Отзыв на доклад}}\\[1em]
    {\large \textbf{Название доклада:} Использование методов подсчёта неопределённости для борьбы с атаками на детекторы машинно-сгенерированного текста}\\[0.5em]
    {\large \textbf{Автор доклада:} Леванов В.Д.}\\[0.5em]
    {\large \textbf{Автор отзыва:} Никитин Артем Анатольевич}\\[2em]
\end{center}

% ==============================
% Секция 1: Аннотация
% ==============================
\section*{Аннотация}
В докладе рассматривается задача надёжного обнаружения машинно-сгенерированного текста в присутствии атак на генеративную
модель и её вывод. Предлагается использовать методы оценки неопределённости (Uncertainty Estimation) в настройках White-box
и Black-box, ранее успешно применявшиеся в NLP-задачах, но мало исследованные для класcификации Human- vs Machine-Generated.
В качестве экспериментальных данных используются логиты контекста Llama-3.1-8B-Instruct на корпусе M4GT-Bench, где метрики
Perplexity и Mean Token Entropy служат признаками для кластеризации и бинарного классификатора. Ожидается, что обученные на 
RAID и M4GT модели продемонстрируют устойчивую точность при различных атаках.

% ==============================
% Секция 2: Критический комментарий
% ==============================
\section*{Критический комментарий}
Доклад ясно формулирует проблему уязвимости AI-детекторов текстов и обоснованно предлагает Uncertainty Estimation как путь 
к повышению робастности. Сильной стороной является практическое использование реализаций на Llama-3.1-8B-Instruct и корпусе 
M4GT-Bench с демонстрацией кластеризации по Perplexity и Mean Token Entropy. Тем не менее, хотелось бы увидеть конкретные 
численные значения точности и ROC-кривые для разных методов оценки неопределённости. В целом работа выглядит актуальной 
для борьбы с дезинформацией и списыванием, и я лично интересуюсь её продолжением на датасете RAID и расширением набора 
методов Uncertainty Estimation.

\end{document}
