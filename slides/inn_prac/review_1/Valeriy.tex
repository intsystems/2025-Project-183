\documentclass[11pt]{article}

\usepackage[utf8]{inputenc}
\usepackage[T2A]{fontenc}
\usepackage[russian]{babel}
\usepackage{amsmath,amssymb}

\begin{document}

% ==============================
% Заголовок и идентификация отзыва
% ==============================
\begin{center}
    {\Large \textbf{Отзыв на доклад}}\\[1em]
    {\large \textbf{Название доклада:} Использование методов подсчёта неопределённости для борьбы с атаками на детекторы машинно-сгенерированного текста}\\[0.5em]
    {\large \textbf{Автор доклада:} Леванов Валерий}\\[0.5em]
    {\large \textbf{Автор отзыва:} Никитин Артем Анатольевич}\\[2em]
\end{center}

% ==============================
% Секция 1: Аннотация
% ==============================
% Требования к аннотации:
% - Объём: от 70 до 150 слов (примерно 450–1050 символов без пробелов).
% - Несколько предложений, раскрывающих суть доклада:
%   * Формулировка задачи.
%   * Научная мотивация.
%   * Обзор предыдущих решений/результатов.
%   * Перспективы: ожидаемые итоги.
%   * Применяемые методы и наличие промежуточных результатов (если есть).
\section*{Аннотация}
В докладе рассматривается необходимость повышения надёжности детекторов, выявляющих машинно-сгенерированный текст.
Автор отмечает, что многие существующие детекторы уязвимы к таким различным атакам, основанным на генерации. В качестве
возможного решения предлагается использовать методы подсчёта неопределённости, которые ранее
показывали хорошие результаты в других задачах обработки естественного языка, но мало исследовались в контексте
классификации "Human vs Machine Generated". Предполагается применение White-box и Black-box подходов к оценке уверенности
и обучение нейронных сетей, устойчивых к манипуляциям с текстом. По итогам исследования ожидается разработка робастного
AI-детектора, способного надёжно отличать рукописные тексты от сгенерированных.

% ==============================
% Секция 2: Критический комментарий
% ==============================
% Требования к критическому комментарию:
% - Объём: от 50 до 150 слов (примерно 350–1050 символов без пробелов).
% - Несколько предложений, учитывающих сильные/слабые стороны.
% - Чем тема интересна лично вам.
\section*{Комментарий}
В докладе чётко сформулирована проблема: многие детекторы, созданные для определения машинной генерации текста, не справляются с
адаптированными атаками. Методы подсчёта неопределённости, на которых был сделан акцент в докладе на семинаре, помогают
оценивать уверенность модели, что может повысить робастность к манипуляциям. Отмечу отдельно, что презентация содержит обзор
смежных решений в NLP и освещает потенциально полезный опыт из компьютерного зрения, что для меня подчеркнуло актуальность 
исследований и наличие существенных продвижений на данный момент. Во время презентации хотелось бы видеть более
детализированный план по экспериментальному сравнению разных методов Uncertainty Estimation. Работа является актуальной в 
текущих реалиях в области борьбы с дезинформацией и повышения академической успеваемости.

\end{document}
