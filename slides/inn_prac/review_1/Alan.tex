\documentclass[11pt]{article}

\usepackage[utf8]{inputenc}
\usepackage[T2A]{fontenc}
\usepackage[russian]{babel}
\usepackage{amsmath,amssymb}

\begin{document}

% ==============================
% Заголовок и идентификация отзыва
% ==============================
\begin{center}
    {\Large \textbf{Отзыв на доклад}}\\[1em]
    {\large \textbf{Название доклада:} Оценки чисел Рамсея для произвольных последовательностей графов}\\[0.5em]
    {\large \textbf{Автор доклада:} Беремкулов Алан}\\[0.5em]
    {\large \textbf{Автор отзыва:} Никитин Артем Анатольевич}\\[2em]
\end{center}

% ==============================
% Секция 1: Аннотация
% ==============================
% Требования к аннотации:
% - Объём: от 70 до 150 слов (примерно 450–1050 символов без пробелов).
% - Текст должен состоять из нескольких предложений.
% Необходимые элементы (при наличии в докладе):
%   * Формулировка задачи.
%   * Научная мотивация (зачем заниматься данной задачей).
%   * Обзор предыдущих результатов.
%   * Перспективы: какие результаты ожидается достичь.
%   * Описание предложенных методов и наличие промежуточных результатов.
\section*{Аннотация}
Доклад посвящён исследованию чисел Рамсея для произвольных последовательностей графов. Проблема заключаетсяв том, что
классическая теория Рамсея остаётся слабо изученной, и точные оценки, например для R(5, 5), неизвестны. Автор вводит
новое определение чисел Рамсея для последовательностей графов, в рамках которого классические числа представляют тривиальный
частный случай. В работе приводятся связанные результаты, в частности оценки, полученные в исследованиях А.М. Райгородского,
а также обсуждаются точные и асимптотические оценки для отдельных классов графов. Ожидается, что дальнейшее применение
предложенного подхода позволит найти числа $R_{\min}$ для графов Джонсона и обнаружить рекурсивные зависимости,
не зависящие от числа независимости графа.

% ==============================
% Секция 2: Критический комментарий
% ==============================
% Требования к критическому комментарию:
% - Объём: от 50 до 150 слов (примерно 350–1050 символов без пробелов).
% - Текст должен состоять из нескольких предложений.
% Обсудите сильные и слабые стороны доклада с точки зрения структуры, презентации и научного содержания.
\section*{Комментарий}
Доклад представляет исследование с конкретной формулировкой проблемы и предложением нового определения чисел Рамсея
для последовательностей графов. Хочу отметить логичность связи с ранее опубликованными оценками, что подчёркивает 
значимость проблемы. Докладчик на семинаре ясно изложил области применения его работы, особенно
в части обоснования необходимости расширения классического понятия чисел Рамсея. Единственным недостатком можно отметить
недостаток подробного обсуждения примения предложенных идеи на графах без дополнительных ограничений. Работа выглядит 
перспективной и актуальной для дальнейшего развития теории Рамсея в дискретной математике, и для меня была интересной в силу 
изучения изложенной теории на курсе Дискретной математики, на котором я решал связанные с данной проблемой задачи.

\end{document}
