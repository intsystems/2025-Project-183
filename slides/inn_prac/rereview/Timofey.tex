\documentclass[11pt]{article}

\usepackage[utf8]{inputenc}
\usepackage[T2A]{fontenc}
\usepackage[russian]{babel}
\usepackage{amsmath,amssymb}
\usepackage[left=2cm,right=2cm,top=2cm,bottom=2cm]{geometry}

\begin{document}

\pagestyle{empty}

\begin{center}
    \subsection*{Ответ на замечание рецензента Кондрашова Тимофея}
\end{center}

\begin{quote}
    \textbf{Рецензент:}
    ``К минусам я бы причислил отсутствие дальнейших шагов и результатов, на которые автор рассчитывает по итогу своей работы.''
\end{quote}

\paragraph{Ответ автора.}
Благодарю за важный комментарий. Действительно, в текущей версии доклада я сконцентрировался на постановке задачи,
описании метода и первых результатах на MNIST/Fashion-MNIST. В окончательной статье я планирую добавить раздел ``Планы и
будущие работы'', в котором подробно освещу следующие этапы:

\begin{itemize}
    \item \textbf{Расширение экспериментов на более сложные датасеты и архитектуры.}
          \begin{itemize}
              \item Проверка критерия ``достаточного размера выборки'' $K$ на CIFAR.
              \item Тестирование на различных моделях, (ConvNet, ResNet).
          \end{itemize}

    \item \textbf{Адаптация метода к потоковым (streaming) данным.}
          \begin{itemize}
              \item Разработка версии алгоритма с динамическим отслеживанием $\Delta_k$ и числом собственных векторов $d$.
          \end{itemize}

    \item \textbf{Уточнение теоретических оценок порога~$K$.}
          \begin{itemize}
              \item На данный момент использую приближение
                    $$\Delta_k \approx \frac{\sigma^4}{4}\Bigl(2\sum_{i=1}^d(\lambda^{(i)}_{k+1}-\lambda^{(i)}_k)^2
                        +\Bigl(\sum_{i=1}^d(\lambda^{(i)}_{k+1}-\lambda^{(i)}_k)\Bigr)^2\Bigr).$$
              \item В дальнейшем планируем вывести более точные неасимптотические оценки скорости сходимости~$\Delta_k\to0$ в
                    зависимости от распределения данных.
          \end{itemize}

    \item \textbf{Визуализация результатов и открытый релиз кода.}
          \begin{itemize}
              \item Дополнительные графики: зависимость точности от значений $\Delta$ функции.
              \item Публикация кода на GitHub (скрипты для Monte Carlo в низкоразмерном подпространстве и расчёта порога $K$).
          \end{itemize}
\end{itemize}

Таким образом, в финальной версии статьи будет раздел ``Планы и будущие работы'', где читатель найдёт полный обзор дальнейших
шагов и ожидаемых результатов. Еще раз благодарю за замечание  --- оно поможет сделать статью более завершённой и
информативной.

\end{document}
