\documentclass[11pt]{article}

\usepackage[utf8]{inputenc}
\usepackage[T2A]{fontenc}
\usepackage[russian]{babel}
\usepackage{amsmath,amssymb}
\usepackage[left=2cm,right=2cm,top=2cm,bottom=2cm]{geometry}

\begin{document}

\pagestyle{empty}

\begin{center}
    \subsection*{Ответ на замечание рецензента Нехорошкова Георгия Сергеевича}
\end{center}

\begin{quote}
    \textbf{Рецензент:}
    ``Вместе с тем, выбранные датасеты (MNIST и Fashion-MNIST) кажутся недостаточно сложными, и было бы интересно увидеть 
    успешное применение метода на более продвинутых данных и архитектурах.''
\end{quote}

\paragraph{Ответ автора.}
Благодарю за высокую оценку теоретической части работы и отзывчивый комментарий по поводу выбора датасетов. Полностью 
согласен, что MNIST и Fashion-MNIST служат лишь отправной точкой для демонстрации моего метода, и использование более 
сложных наборов данных и моделей позволит ещё более убедительно показать его практическую ценность. В окончательной версии 
статьи я планирую:

\begin{enumerate}
    \item \textbf{Эксперименты на CIFAR.}
          Эти датасеты представляют более богатые визуальные и статистические структуры, что позволит проверить, насколько 
          быстро убывает функция $\Delta_k$ при росте объёма выборки в условиях более разнообразных входов.
    \item \textbf{Проверка на современных архитектурах.}
          Я реализую проекцию на субпространство, выделенное по верхним собственным векторам Гессиана, для сетей 
          ResNet и ConvNet. Такой выбор даст представление о том, насколько эффективно метод выявляет ``точку насыщения'' 
          датасета в высокопараметрических пространствах.
    \item \textbf{Промежуточные результаты.}
          Мои предварительные эксперименты на MNIST с MLP показывают, что порог достаточного размера выборки 
          $K$ определяется в районе 10000 изображений, что согласуется с эмпирической практикой остановки обучения 
          на этих датасетах. Эти данные будут включены в раздел ``Эксперименты'' вместе с соответствующими графиками 
          зависимости $\Delta_k$ от $k$.
\end{enumerate}

Таким образом, расширение набора испытаний на более сложные данные и современные архитектуры не только подтвердит 
универсальность предложенного подхода, но и позволит практикам получить конкретные рекомендации по выбору размера обучающей 
выборки в приложениях с высокой размерностью.

\end{document}
