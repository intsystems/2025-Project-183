\documentclass[11pt]{article}

\usepackage[utf8]{inputenc}
\usepackage[T2A]{fontenc}
\usepackage[russian]{babel}
\usepackage{amsmath,amssymb}

\begin{document}

% ==============================
% Заголовок и идентификация отзыва
% ==============================
\begin{center}
    {\Large \textbf{Отзыв на доклад}}\\[1em]
    {\large \textbf{Название доклада:} Локализация по графу 3D-сцены}\\[0.5em]
    {\large \textbf{Автор доклада:} Кондрашов Тимофей}\\[0.5em]
    {\large \textbf{Автор отзыва:} Никитин Артем Анатольевич}\\[2em]
\end{center}

% ==============================
% Секция 1: Аннотация
% ==============================
% Требования к аннотации:
% - Объём: от 70 до 150 слов (примерно 450–1050 символов без пробелов).
% - Текст должен состоять из нескольких предложений.
% Необходимые элементы (при наличии в докладе):
%   * Формулировка задачи.
%   * Научная мотивация.
%   * Обзор предыдущих результатов (если есть).
%   * Перспективы и ожидаемые результаты.
%   * Используемые методы/планы и наличие промежуточных результатов (если есть).
\section*{Аннотация}
В докладе рассматривается задача локализации робота по данным с RGBD-сенсоров в помещениях, что является одним из ключевых 
направлений при создании автономных систем. Ключевая идея — использовать графы сцен, где вершины соответствуют объектам, 
а рёбра отражают пространственные либо семантические связи. Такой способ представления данных меньше 
зависит от изменений освещения, точки обзора и динамического окружения. Автор отмечает, что существующих исследований, 
посвящённых графам 3D-сцен в задаче локализации, крайне мало, и предлагает изучать подход на базе методики SceneGraphLoc 
и датасета 3RScan. Ожидается, что итоговый результат будет включать нейронную сеть, способную работать только с RGBD-данными 
и эффективно решать задачу локализации робота на основе графов сцен.

% ==============================
% Секция 2: Критический комментарий
% ==============================
% Требования к критическому комментарию:
% - Объём: от 50 до 150 слов (примерно 350–1050 символов без пробелов).
% - Текст должен состоять из нескольких предложений.
% Укажите сильные и слабые стороны доклада, его структуру и научное содержание, 
% а также, чем тема интересна лично вам.
\section*{Комментарий}
Доклад весьма актуален, так как задача локализации робота остается одной из важнейших в робототехнике. Автор 
акцентируетт внимание именно графам 3D-сцен, позволяющим хранить информацию о топологии и семантике объектов, что 
потенциально снижает зависимость от внешних факторов. Презентация получилась структурированной, хотя хотелось 
бы услышать более детальные результаты предварительных тестов. Тем не менее, сравнение с аналогичными методами, упомянутыми 
в докладе (например, SceneGraphLoc), даёт представление о том, как будет организована эксперименты по оценке нейронной сети. 
Лично меня тема привлекает своей связью с компьютерным зрением и возможностями применения в широком спектре автономных систем.

\end{document}
