\documentclass[11pt]{article}

\usepackage[utf8]{inputenc}
\usepackage[T2A]{fontenc}
\usepackage[russian]{babel}
\usepackage{amsmath,amssymb}
\usepackage[left=2cm,right=2cm,top=2cm,bottom=2cm]{geometry}

\begin{document}

\pagestyle{empty}

\begin{center}
    \subsection*{Ответ на замечание рецензента Кравацкого Алексея Юрьевича}
\end{center}

\begin{quote}
    \textbf{Рецензент:}
    ``I was particularly interested in this presentation because the idea was not to improve accuracy at the expense of even
    higher computational cost, but rather to preserve accuracy while reducing the cost.''
\end{quote}

\paragraph{Ответ.}
Благодарю за высокую оценку фокуса работы на оптимизации соотношения ``точность --- вычислительные ресурсы''. Действительно,
ключевая цель моего метода — определить порог объёма данных $K$, после которого дальнейшее наращивание выборки приводит к
пренебрежимо малым изменениям ландшафта потерь, позволяя экономить ресурсы без потери точности.

\begin{quote}
    \textbf{Рецензент:}
    ``I like the style of the presentation: it is informative and neat.''
\end{quote}

\paragraph{Ответ.}
Рад, что оформление и структура доклада оказались понятными и информативными --- я старался сделать материалы понятными
для аудитории.

\begin{quote}
    \textbf{Рецензент:}
    ``I would like to learn more about plans regarding both theory and experiments.''
\end{quote}

\paragraph{Ответ.}
В финальной версии статьи я введу раздел ``Планы и будущие работы'', где опишу:
\begin{itemize}
    \item Расширение теоретических выкладок: уточнение неасимптотических оценок сходимости $\Delta_k$ в зависимости от
          свойств распределения данных и архитектуры сети.
    \item Проведение новых экспериментов на CIFAR и современных архитектурах (ResNet, ConvNet).
\end{itemize}

\begin{quote}
    \textbf{Рецензент:}
    ``Verification of the theoretical bounds was mentioned in the “results”, but the bounds were not presented, which is
    disappointing, as they must convey the gist of the method.''
\end{quote}

\paragraph{Ответ.}
Действительно, в устной версии доклада я лишь упомянул проверку теоретической формулы
$$\Delta_k \approx \frac{\sigma^4}{4}\Bigl(2\sum_{i=1}^d(\lambda^{(i)}_{k+1}-\lambda^{(i)}_k)^2
    +\Bigl(\sum_{i=1}^d(\lambda^{(i)}_{k+1}-\lambda^{(i)}_k)\Bigr)^2\Bigr),$$
но не показал соответствующие графики. В окончательном тексте я включу графическое сравнение теоретической кривой и
эмпирических значений $\Delta_k$.

\begin{quote}
    \textbf{Рецензент:}
    ``I would like to learn more about applications of the method to different architectures. The experiment has been already
    conducted on MNIST and Fashion-MNIST, which is commendable, but many a conference reviewer would be particularly
    interested in possibility of application of the method to transformers and other complex architectures.''
\end{quote}

\paragraph{Ответ.}
Полностью разделяю интерес к применению метода на современных глубоких моделях. В разделе ``Эксперименты'' статьи я представлю
предварительные результаты для ResNet и ConvNet на CIFAR,
демонстрируя, что ``активное'' подпространство, выделенное по первым $d$ собственным векторам Гессиана, устойчиво к
усложнению архитектуры.

Эти дополнения сделают статью более убедительной для аудитории, работающей с современными архитектурами.

\end{document}
