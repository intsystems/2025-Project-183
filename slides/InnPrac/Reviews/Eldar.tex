\documentclass[11pt]{article}

\usepackage[utf8]{inputenc}
\usepackage[T2A]{fontenc}
\usepackage[russian]{babel}
\usepackage{amsmath,amssymb}

\begin{document}

% ==============================
% Заголовок и идентификация отзыва
% ==============================
\begin{center}
    {\Large \textbf{Отзыв на доклад}}\\[1em]
    {\large \textbf{Название доклада:} Анализ воспроизводимости значимости статических характеристик водосборов при моделировании речного стока и уровня воды методами машинного обучения}\\[0.5em]    {\large \textbf{Авторы доклада:} Хузин Эльдар}\\[0.5em]
    {\large \textbf{Автор отзыва:} Никитин Артем Анатольевич}\\[2em]
\end{center}

% ==============================
% Секция 1: Аннотация
% ==============================
% Краткое изложение основных идей доклада:
% - Формулировка задачи: оптимизация прогнозирования гидрологических процессов с учётом ограниченности данных и вычислительных ресурсов.
% - Научная мотивация: необходимость точного и экономически эффективного моделирования водных ресурсов для предупреждения наводнений и повышения управления водными объектами.
% - Обзор существующих решений и их недостатков: физико-обоснованные модели требуют огромных ресурсов, а data-driven методы зачастую не используют дополнительные физико-географические признаки.
% - Предлагаемое решение: сравнительный анализ различных data-driven подходов (Random Forest, LSTM, трансформеры) с включением физико-географических атрибутов, оценка моделей по метрикам RMSE, MAE, MAPE, NSE.
% - Ожидаемые результаты: выявление оптимальной комбинации модели и набора признаков, позволяющей снизить вычислительные затраты и повысить точность прогнозов.
\section*{Аннотация}
Доклад посвящён сравнительному анализу data-driven подходов для прогнозирования речного стока и уровня воды. Основная задача ---
определить, в какой мере использование дополнительных данных, включая физико-географические атрибуты, способствует улучшению
метрик моделей при ограниченных вычислительных ресурсах. Автор рассматривает
современные нейросетевые архитектуры (LSTM, трансформеры) в контексте их способности учитывать особенности гидрологических
процессов. Предлагаемый подход основан на оценке моделей по различным метрикам, с целью выявления оптимальной,
которая бы сочетала высокую точность прогнозов с разумными вычислительными затратами. Результатом
исследования станет практическая рекомендация по выбору модели для гидрологического прогнозирования.

% ==============================
% Секция 2: Критический комментарий
% ==============================
% Оцените сильные и слабые стороны доклада:
% - Сильные стороны: комплексный анализ, чётко сформулированная задача, использование разнообразных моделей и метрик, практическая направленность.
% - Недостатки: отдельные разделы, касающиеся описания экспериментов, могли бы быть дополнены подробными результатами, что способствовало бы лучшему пониманию практической ценности подхода.
% - Личный интерес: тема актуальна для областей, связанных с управлением водными ресурсами и предотвращением стихийных бедствий, что вызывает особый интерес благодаря практическому применению.
\section*{Комментарий}
В работе представлен подход к решению задачи гидрологического прогнозирования, что особенно важно для оптимизации
использования данных и вычислительных ресурсов. Преимуществом работы я бы отметил сравнительный анализ различных моделей,
а также интеграцию физико-географических признаков для повышения точности прогнозов. Автор использует различные
метрики для оценки качества на реальных данных, что позволяет увидеть практическую пользу подхода. Однако хотелось бы получить
более детальное сравнение с существующими методами и описание области применения в гидрологии: не до конца ясна вычислительная
затратность метода. Меня тема привлекает практической значимостью для предотвращения наводнений, что подчеркнуло для меня
актуальность предложенного исследования.

\end{document}
